\chapter{Úvod}
Internet je zdrojem nepřeberného množství veřejně dostupných dat. Tato data jsou často sice použitelná, z hlediska strojového zpracování jsou ale velmi špatně zpracovatelná a tedy nepřipravená pro další použití. Pro další využití těchto dat je nutné je určitým způsobem zpracovat -- unifikovat jejich podobu a formát, zasadit je do kontextu ostatních dat, případně spojit více zdrojů.

Jedním z nejznámějších a nejrozsáhlejších veřejně přístupných zdrojů dat je v současné době Wikipedie. Tento projekt využívá odbornou veřejnost pro tvorbu obsahu, který je pak poskytován formou webové encyklopedie. Vzhledem k množství tohoto obsahu, který je navíc částečně strukturovaný a tedy lépe strojově čitelný, to činí Wikipedii vhodným zdrojem dat. Cílem této práce je zanalyzovat dostupné nástroje pro práci s~Wikipedií a~s~jejich pomocí následně navrhnout a~vytvořit aplikaci, která dokáže vhodným způsobem tato data získat. Preferovaná pak budou data o historických událostech, obecně tedy články mající časový rozsah, pomocí něhož mohou být následně zasazeny do kontextu s~ostatními událostmi.

Význam takto získaných dat ale vždy záleží až na způsobu jejich využití. Pro hlubší a~rychlejší pochopení problému je nutné získaná data uživateli určitým způsobem reprezentovat. Spíše než prostý výčet dat je vhodnější využít vizualizaci, která působí z pohledu uživatele přehledněji a přívětivěji. V případě historických událostí se na první pohled jeví jako vhodný způsob reprezentace časová osa. Je nutné vytvořit nástroj, který veřejně přístupná data dokáže využít a připravit je pro potřeby vizualizace.

Aplikace, která bude v rámci této práce vytvořena, tedy umožní snazší a rychlejší přípravu dat o historických událostech z Wikipedie. Uživateli dá možnost vybrat jím preferované články. Ty aplikace vhodně analyzuje a po zpracování příslušné části Wikipedie se pokusí ostatním článkům přiřadit ohodnocení a uživateli prezentovat články jevící se jako nejrelevantnější.

Výběr platformy, s kterou bude nástroj pracovat, bude záležet na výsledku analýzy, jejímž cílem bude především shrnout výhody a nevýhody jednotlivých přístupů k Wikipedii. Tento výběr bude učiněn s ohledem na jednoduchost implementace (z programovacího hlediska), stabilitu a dostupnost řešení. Platforma musí být také aktuální a aktualizovaná, aby umožnila dlouhodobou funkčnost vytvářeného nástroje.

Výstupem aplikace bude graf, jehož vrcholy budou představovat jednotlivé historické události, které byly do výběru zařazeny. Tyto vrcholy budou obsahovat dodatečné informace, jako název, stručný popis, obrázek a další položky. Hrany mezi nimi budou značit určitý vztah mezi články. Tento výstup bude následně zkontrolován a ohodnocen především s ohledem na přesnost a úplnost -- u některých témat lze jednoduše předvídat, jaké články by měly být navrhovány a které jsou naprosto nevhodné, obecné či nepříbuzné.