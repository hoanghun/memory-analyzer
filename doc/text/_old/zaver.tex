\chapter{Závěr}
Na základě provedené analýzy byl pro řešení problému použit systém rozšíření webového prohlížeče Google Chrome, který se pro tento typ úlohy jevil jako nejvhodnější. Pro implementaci nástroje byl zvolen programovací jazyk TypeScript, který programátorovi poskytuje pohodlnější prostředí než JavaScript, jazyk běžně používaný v prohlížeči Google Chrome. Tento jazyk umožňuje používat konstrukce běžné v jiných programovacích jazycích, jako je Java, ale na druhou stranu není jeho použití vyžadováno a je stále možné využít JavaScript. V případě pokračování ve vývoji nástroje tak může příští autor zvolit mezi těmito dvěma jazyky dle jeho preference.

Z hlediska použitelnosti a dostupnosti byla jako zdroj dat vybrána online verze Wikipedie, díky čemuž může nástroj pracovat pouze v rámci prohlížeče bez nutnosti instalovat jakýkoliv další software. Data jsou také vždy aktuální a díky existujícímu \zk{zkAPI} je možné dynamicky získávat informace, které prosté zobrazení obsahu článku neposkytne. Tyto dodatečné informace byly využity také při tvorbě systému ohodnocení, který je vytvořen takovým způsobem, aby jeho jednotlivé složky bylo možné měnit, přidávat či upravovat, případně měnit jejich prioritu.

Možnost dalšího pokračování v práci je především v oblasti modulů pro zpracování infoboxů. Nebylo účelem práce ani záměrem pokrýt velké množství typů článků, ale spíše poskytnout jednoduchý prostředek pro jejich tvorbu. Toho bylo docíleno právě modulární koncepcí, čímž je umožněno jednoduché přidání dalších typů bez nutnosti zasáhnout do~zbytku aplikace. Rovněž je stejným způsobem možné přidávat další jazyky a není tedy nutné se omezit na články v angličtině, na kterou byla práce zaměřena. Na menším vzorku předpřipravených modulů byly ukázány jejich možnosti.

Vytvořený nástroj v současném stavu splňuje podmínky a nároky kladené v zadání a~v~úvodu. Cílem bylo vytvořit prostředek pro zpracovávání historických událostí tak, aby mohly být rychle a jednoduše uvedeny do kontextu s ostatními prostřednictvím časové osy -- což bylo dle testování splněno a poskytuje uspokojivé a očekávatelné výsledky. Testování rovněž ukázalo, že v případě článků, jejichž zpracování je v modulech infoboxů podporováno, dokáže spolehlivě rozeznat důležitá data a spojení mezi jednotlivými články. V~některých případech rozeznávání sice selže, což ale není nutně chyba nástroje, ale spíše vlastnost Wikipedie plynoucí z její podstaty -- původu obsahu. Řešením pro tato selhání by bylo zavést při zpracovávání některé techniky z oblasti \zk{zkNLP}, případně striktnější dodržování syntaxe na straně Wikipedie.
